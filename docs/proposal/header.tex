\usepackage{graphicx}
\usepackage{subfigure}
\usepackage{wrapfig}
%\usepackage{floatflt} %don't think we ever use it, and it isn't included in current tex distributions
%\usepackage{epsfig}
\usepackage{color}
\usepackage{url} %hyperref includes url functionality
\usepackage{listings}
\usepackage{pstricks}
\usepackage{multirow}
%\usepackage[toc,page]{appendix}  % to add appendices using "\begin{appendices} ...  \end{appendices}"
				 % or using things like "\appendix, \appendixpage, \addappheadtotoc"

\usepackage{balance} % add the command \balance if you want to balance the columns on the 
                     % last page. 


\usepackage{amsmath, amssymb}	% From American Mathematical Society, for theorems and proofs
\ifdefined\proof
\else
\ifdefined\theoremstyle
\else
\usepackage{amsthm}
			% See http://www.cs.hmc.edu/qref/latex/qref/#AMS
			%TE: Gives errors when included (\proof already defined?)
\fi
\fi

%\usepackage[ps2pdf,breaklinks,pdfpagemode=none]{hyperref}
%\usepackage{breakurl} %allows links in hyperref to split lines
\usepackage[12hr,nodate]{datetime}
%\usepackage{psfrag} %allow text substitution in postscript:
% enclose the \includegraphics command with curly braces and add a \psfrag command for 
% each piece of text which you would like to replace in that figure, 
% with the corresponding Latex text following it.
%
% {
%   \psfrag{alpha}{$\alpha$}
%   \psfrag{eqn1}{$\int_0^\infty\frac{x^2yz}{x^\alpha}{\rm d}x$}
%   \includegraphics{figure.eps}
% }
%


\newcommand{\oldstuff}[1]{}
\newcommand{\optional}[1]{}
\newcommand{\consider}[1]{}
\newcommand{\moved}[1]{}
\newcommand{\comments}[1]{}
\newcommand{\comment}[1]{}
\newcommand{\TODO}[1]{}

\newcommand{\eg}{{e.g., }}
\newcommand{\Eg}{{E.g., }}
\newcommand{\ie}{{i.e., }}
\newcommand{\Ie}{{I.e., }}
\ifdefined\etal
\else
  \newcommand{\etal}{{et al. }} % et is a word, al. is an abbreviation (et alia)
\fi

\newcommand{\important}[1]{\emph{\textbf{#1}}}
\newcommand{\term}[1]{{\textsf{\textbf{\footnotesize{#1}}}}}
\newcommand{\leading}[1]{\emph{#1}}

\ifdefined\theorem
\else
\newtheorem{theorem}{Theorem}[section]
\newtheorem{conjecture}[theorem]{Conjecture}
\newtheorem{corollary}[theorem]{Corollary}
\newtheorem{proposition}[theorem]{Proposition}
\newtheorem{lemma}[theorem]{Lemma}
%\newdef{definition}[theorem]{Definition}
%\newdef{remark}[theorem]{Remark}
\fi

% \cramplist and \cramp are commands you can use within a list (bulleted, itemized, whatever), and 
% bullets is an environment you can use in place of itemize.

%%%
%%%  Cramped lists, for occasionally squeezing a bit too much
%%%%     onto a slide
\newcommand{\cramplist}{
        \setlength{\itemsep}{0in}
        \setlength{\partopsep}{0in}
        \setlength{\topsep}{0in}}
\newcommand{\cramp}{\setlength{\parskip}{.5\parskip}}

%
%  Bulleted lists, almost flush with left margin
%
\newenvironment{bullets}%
{\begin{list}{$\bullet$}{\setlength{\leftmargin}{1.5ex}%  originally 1 ex
\setlength{\itemindent}{.5ex}}}% originally .5
{\end{list}}

%
% moving figures and tables 
%
\optional{
        \usepackage{figcaps}    % Move all figures and tables to the end of file
        \printfigures           % Print these figures
        % \figmarkon            % Leave a mark in text to show where to insert figures or tables
}
